\documentclass[10pt]{article}  

%%%%%%%% PREÁMBULO %%%%%%%%%%%%
\title{Reporte de Laboratorio}
\usepackage[spanish]{babel} %Indica que escribiermos en español
\usepackage[utf8]{inputenc} %Indica qué codificación se está usando ISO-8859-1(latin1)  o utf8  
\usepackage{amsmath} % Comandos extras para matemáticas (cajas para ecuaciones,
% etc)
\usepackage{amssymb} % Simbolos matematicos (por lo tanto)
\usepackage{longtable} %agregadom para hacer tablas
\usepackage{xltabular}
\usepackage{tabularray}
\usepackage{graphicx} % Incluir imágenes en LaTeX
\usepackage{color} % Para colorear texto
\usepackage{subfigure} % subfiguras
\usepackage{float} %Podemos usar el especificador [H] en las figuras para que se
% queden donde queramos
\usepackage{capt-of} % Permite usar etiquetas fuera de elementos flotantes
% (etiquetas de figuras)
\usepackage{sidecap} % Para poner el texto de las imágenes al lado
	\sidecaptionvpos{figure}{c} % Para que el texto se alinie al centro vertical
\usepackage{caption} % Para poder quitar numeracion de figuras
\usepackage{commath} % funcionalidades extras para diferenciales, integrales,
% etc (\od, \dif, etc)
\usepackage{cancel} % para cancelar expresiones (\cancelto{0}{x})
 
\usepackage{anysize} 					% Para personalizar el ancho de  los márgenes
\marginsize{2cm}{2cm}{2cm}{2cm} % Izquierda, derecha, arriba, abajo

\usepackage{appendix}
\renewcommand{\appendixname}{Apéndices}
\renewcommand{\appendixtocname}{Apéndices}
\renewcommand{\appendixpagename}{Apéndices} 
% Para que las referencias sean hipervínculos a las figuras o ecuaciones y
% aparezcan en color
\usepackage[colorlinks=true,plainpages=true,citecolor=blue,linkcolor=blue]{hyperref}
%\usepackage{hyperref} 
% Para agregar encabezado y pie de página
\usepackage{fancyhdr} 
\pagestyle{fancy}
\fancyhf{}
\fancyhead[L]{\footnotesize UNI} %encabezado izquierda
\fancyhead[R]{\footnotesize Fac. de Ciencias}   % dereecha
\fancyfoot[R]{\footnotesize Reporte}  % Pie derecha
\fancyfoot[C]{\thepage}  % centro
\fancyfoot[L]{\footnotesize Reporte de conteo y medicion.}  %izquierda
\renewcommand{\footrulewidth}{0.4pt}


\usepackage{listings} % Para usar código fuente
\definecolor{dkgreen}{rgb}{0,0.6,0} % Definimos colores para usar en el código
\definecolor{gray}{rgb}{0.5,0.5,0.5} 
% configuración para el lenguaje que queramos utilizar
\lstset{language=Matlab,
   keywords={break,case,catch,continue,else,elseif,end,for,function,
      global,if,otherwise,persistent,return,switch,try,while},
   basicstyle=\ttfamily,
   keywordstyle=\color{blue},
   commentstyle=\color{red},
   stringstyle=\color{dkgreen},
   numbers=left,
   numberstyle=\tiny\color{gray},
   stepnumber=1,
   numbersep=10pt,
   backgroundcolor=\color{white},
   tabsize=4,
   showspaces=false,
   showstringspaces=false}

\newcommand{\sen}{\operatorname{\sen}}	% Definimos el comando \sen para el seno
%en español

\title{Plantilla para Reportes IMEC-UTB}
% Basada en la plantilla para reportes UPIITA de  Overleaf

%%%%%%%% TERMINA PREÁMBULO %%%%%%%%%%%%

\begin{document}

%%%%%%%%%%%%%%%%%%%%%%%%%%%%%%%%%% PORTADA %%%%%%%%%%%%%%%%%%%%%%%%%%%%%%%%%%%%%%%%%%%%
																					%%%
\begin{center}																		%%%
\newcommand{\HRule}{\rule{\linewidth}{0.5mm}}									%%%\left
 																					%%%

\hspace{0.9cm}

\textsc{\huge Universidad Nacional de Ingeniería }\\[0.8cm]
			
\textsc{\LARGE Facultad de Ciencias}

\vspace{0.6cm}


\includegraphics[scale = 0.15]{Imagenes/UNI.png}

													 								%%%
\vspace*{0.6cm}								%%%
																					%%%	



\begin{minipage}{0.9\textwidth} 
\begin{center}																					%%%
\textsc{\LARGE  Laboratorio de Física BFI01 A\\[0.7cm] Reporte 03}
\end{center}
\end{minipage}\\[0.3cm]
%%%
    																				%%%
 			\vspace*{0.4cm}																		%%%
																					%%%
\HRule \\[0.5cm]																	%%%
{ \huge \bfseries Segunda Ley de Newton}\\[0.2cm]	%%%
 																					%%%
\HRule \\[0.9cm]																	%%%
 																				%%%
																					%%%
\begin{minipage}{0.46\textwidth}													%%%
\begin{flushleft} \large															%%%

% Aqui a continuación pongan los nombres de los integrantes
\emph{El grupo conformado por:}\\[2mm]
Llactahuaman Quispe Benjamin\\20232268G \\[1mm]
Cortez Núñez Christian\\20232203B\\[1mm]
Rocca Cruz Axel\\20234046A\\[1mm]
 

%%%
			%\vspace*{2cm}	
            													%%%
										 						%%%
\end{flushleft}																		%%%
\end{minipage}		
																%%%
\begin{minipage}{0.52\textwidth}		
\vspace{-2.6cm}											%%%
\begin{flushright} \large															%%%
\emph{Profesor:}\\[2mm]																	%%%
 Marcelino Eusebio Davila Ingaruca\\
\end{flushright}																	%%%
\end{minipage}	
\vspace*{1cm}
%\begin{flushleft}
 	
%\end{flushleft}
%%%																	%%%															
\vspace{-3.3cm}	
\begin{flushright}	
\large
\emph{Fecha de la práctica:}\\
\vspace{0.1cm}
3 de mayo de 2023
\end{flushright}
 \vspace{2.7cm}

\large{\today}\\
\vspace{0.5cm}

{ \Large\bfseries 2023-1}
										 			
\end{center}							 											
																					
\newpage																		
%%%%%%%%%%%%%%%%%%%% TERMINA PORTADA %%%%%%%%%%%%%%%%%%%%%%%%%%%%%%%%

\tableofcontents 

\newpage

\begin{center}
    \textbf{\huge Segunda Ley de Newton}
\end{center}

\section{Objetivos}
\subsection{Competencias Generales}
\begin{itemize}
    \item Verificar experimentalmente la segunda ley de Newton
\end{itemize} 
\subsection{Competencias Específicas}
\begin{itemize}
    \item Calibrar dos resortes para hallar su constante elástica y con ello hallar la fuerza que estos aplican sobre un disco.
    \item Hallar la aceleración a partir de los ticks marcados por el disco sobre el papel donde realiza su trayectoria.
\end{itemize}

\section{Datos obtenidos}

\newpage

\section{Cálculos y resultados}


\section{Observaciones experimentales}

\section{Cuestionario}
\begin{itemize}
    \item Presente la curva de calibración de cada resorte y calcule la constante elástica de cada resorte
    
    \item Determine en newton el módulo de la fuerza resultante que los resortes ejercieron sobre el disco en 3 puntos de la trayectoria. Para obtener mejores los resultados es preferible trabajar con puntos donde la curvatura de la trayectoria es mayor
    
    \item Determine aproximadamente el vector velocidad instantánea en los instantes intermedios anterior y posterior de los puntos elegidos, es decir si usted eligió el punto correspondiente a t = 8 ticks para hallar la fuerza, halle la velocidad en los instantes t = 7,5 ticks y t = 8,5 ticks. Para ello efectúe la siguiente operación:
    \begin{equation*}
        v(t = 7,5~ticks)= \frac{r(t = 8) - r(t = 7)}{1~tick} 
    \end{equation*}
    
    \item Determine la aceleración instantánea en el punto elegido. Si ese punto es el correspondiente al instante t = 8 tick.
    \begin{equation*}
        a(t = 8~ticks)= \frac{v(t = 8,5) - v(t = 7,5)}{1~tick} 
    \end{equation*}
    
    \item Repita el mismo criterio de los 2 items anteriores sobre los otros dos puntos elegidos.
    \item Determine la relación entre los módulos del vector fuerza y el vector aceleración en cada instante considerado
\end{itemize}

\begin{table}[H]
\centering
\begin{tblr}{
  cells = {c},
  hline{1,5} = {-}{0.08em},
  hline{2} = {-}{},
}
Instante (tick) & \textbar{}a\textbar{} (m/s$^{2}$) & \textbar{}F\textbar{} (N) & \textbar{}$\frac{F}{a}$\textbar{} (Kg) \\
                  &                                &                             &                                \\
                  &                                &                             &                                \\
                  &                                &                             &                                
\end{tblr}
\end{table}


\section{Conclusiones}


\section{Bibliografía}
\begin{itemize}
    \item  Serway. F\'isica. Editorial McGraw-Hill (1992).
    \item Tipler. Física. Editorial Revert\'e (1994).
    \item Taylor, J.R. (2014) Introducci\'on al Análisis de errores - reverte, Taylor 2 Muestra.pdf. Disponible en: https://www.reverte.com/media/reverte/files/book-attachment-3746.pdf (Accessed: April 14, 2023).

\end{itemize}
\end{document}