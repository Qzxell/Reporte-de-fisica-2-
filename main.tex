\documentclass[10pt]{article}  

%%%%%%%% PREÁMBULO %%%%%%%%%%%%
\title{Reporte de Laboratorio}
\usepackage[spanish]{babel} %Indica que escribiermos en español
\usepackage[utf8]{inputenc} %Indica qué codificación se está usando ISO-8859-1(latin1)  o utf8  
\usepackage{amsmath} % Comandos extras para matemáticas (cajas para ecuaciones,
% etc)
\usepackage{amssymb} % Simbolos matematicos (por lo tanto)
\usepackage{longtable} %agregadom para hacer tablas
\usepackage{xltabular}
\usepackage{graphicx} % Incluir imágenes en LaTeX
\usepackage{color} % Para colorear texto
\usepackage{subfigure} % subfiguras
\usepackage{float} %Podemos usar el especificador [H] en las figuras para que se
% queden donde queramos
\usepackage{capt-of} % Permite usar etiquetas fuera de elementos flotantes
% (etiquetas de figuras)
\usepackage{sidecap} % Para poner el texto de las imágenes al lado
	\sidecaptionvpos{figure}{c} % Para que el texto se alinie al centro vertical
\usepackage{caption} % Para poder quitar numeracion de figuras
\usepackage{commath} % funcionalidades extras para diferenciales, integrales,
% etc (\od, \dif, etc)
\usepackage{cancel} % para cancelar expresiones (\cancelto{0}{x})
 
\usepackage{anysize} 					% Para personalizar el ancho de  los márgenes
\marginsize{2cm}{2cm}{2cm}{2cm} % Izquierda, derecha, arriba, abajo

\usepackage{appendix}
\renewcommand{\appendixname}{Apéndices}
\renewcommand{\appendixtocname}{Apéndices}
\renewcommand{\appendixpagename}{Apéndices} 
% Para que las referencias sean hipervínculos a las figuras o ecuaciones y
% aparezcan en color
\usepackage[colorlinks=true,plainpages=true,citecolor=blue,linkcolor=blue]{hyperref}
%\usepackage{hyperref} 
% Para agregar encabezado y pie de página
\usepackage{fancyhdr} 
\pagestyle{fancy}
\fancyhf{}
\fancyhead[L]{\footnotesize UNI} %encabezado izquierda
\fancyhead[R]{\footnotesize Fac. de Ciencias}   % dereecha
\fancyfoot[R]{\footnotesize Reporte}  % Pie derecha
\fancyfoot[C]{\thepage}  % centro
\fancyfoot[L]{\footnotesize Reporte de conteo y medicion.}  %izquierda
\renewcommand{\footrulewidth}{0.4pt}


\usepackage{listings} % Para usar código fuente
\definecolor{dkgreen}{rgb}{0,0.6,0} % Definimos colores para usar en el código
\definecolor{gray}{rgb}{0.5,0.5,0.5} 
% configuración para el lenguaje que queramos utilizar
\lstset{language=Matlab,
   keywords={break,case,catch,continue,else,elseif,end,for,function,
      global,if,otherwise,persistent,return,switch,try,while},
   basicstyle=\ttfamily,
   keywordstyle=\color{blue},
   commentstyle=\color{red},
   stringstyle=\color{dkgreen},
   numbers=left,
   numberstyle=\tiny\color{gray},
   stepnumber=1,
   numbersep=10pt,
   backgroundcolor=\color{white},
   tabsize=4,
   showspaces=false,
   showstringspaces=false}

\newcommand{\sen}{\operatorname{\sen}}	% Definimos el comando \sen para el seno
%en español

\title{Plantilla para Reportes IMEC-UTB}
% Basada en la plantilla para reportes UPIITA de  Overleaf

%%%%%%%% TERMINA PREÁMBULO %%%%%%%%%%%%

\begin{document}

%%%%%%%%%%%%%%%%%%%%%%%%%%%%%%%%%% PORTADA %%%%%%%%%%%%%%%%%%%%%%%%%%%%%%%%%%%%%%%%%%%%
																					%%%
\begin{center}																		%%%
\newcommand{\HRule}{\rule{\linewidth}{0.5mm}}									%%%\left
 																					%%%
\begin{minipage}{0.48\textwidth} \begin{flushleft}
\includegraphics[scale = 0.9]{Imagenes/UNI_2.png}
\end{flushleft}\end{minipage}
\begin{minipage}{0.48\textwidth} \begin{flushright}
\includegraphics[scale = 0.075]{Imagenes/UNI.png}
\end{flushright}\end{minipage}

													 								%%%
\vspace*{1.0cm}								%%%
																					%%%	
\textsc{\huge Universidad Nacional de Ingenería }\\[1 cm]	
{ \huge  Laboratorio de Física N°2}\\


%%%
    																				%%%
 			\vspace*{1cm}																		%%%
																					%%%
\HRule \\																%%%

{\LARGE \bfseries Velocidad y aceleración instantánea\\[0.1cm] en el movimiento rectilíneo}\\[0.2cm]	%%%
 																					%%%
\HRule \\[1cm]														
\begin{minipage}{0.56\textwidth}													%%%
\begin{flushleft} \large															%%%

% Aqui a continuación pongan los nombres de los integrantes
\emph{El grupo conformado por:}\\[0.2cm]	
    Rocca Cruz Axel\\20234046A\\[0.1cm]
    
    Llactahuaman Quispe Benjamin\\
    20232268G\\[0.1cm]
    Cortez Núñez Christian\\
    20232203B

 


%%%
			%\vspace*{2cm}	
            													%%%
										 						%%%
\end{flushleft}																		%%%
\end{minipage}		
																%%%
\begin{minipage}{0.42\textwidth}		
\vspace{-0.6cm}											%%%
\begin{flushright} \large															%%%
\emph{Profesor:} \\[0.2cm]																	%%%
Brocca Pobes Manuel Enrique%%%
\end{flushright}																	%%%
\end{minipage}	
\vspace*{1cm}
%\begin{flushleft}
 	
%\end{flushleft}
%%%
 		\flushleft{\textbf{\Large Física BFI01 - A}	}\\																		%%%
\vspace{1cm} 																				
\begin{center}	
{\large \today}																	%%%
 			\end{center}												  						
\end{center}							 											
																					
\newpage																		
%%%%%%%%%%%%%%%%%%%% TERMINA PORTADA %%%%%%%%%%%%%%%%%%%%%%%%%%%%%%%%

\tableofcontents 

\newpage

\section{Objetivos}
\subsection{Objetivos generales}
Determinar experimentalmente la velocidad instantanea y la aceleracion de la rueda de Maxwell.
\subsection{objetivos espec\'ificos}
\begin{itemize}
    \item Ordenar los datos obtenidos en una tabla.
    \item Obtener distintas comparaciones y sacar conclusiones de los experimentos.
\end{itemize}
\section{C\'alculos}
\section{Resultados}
\section{Observaciones Experimentales}
\section{Conclusiones}
\begin{itemize}
    \item Podemos notar de los datos analizados varian un poco esto de debe a que al momento de tomar medidas siempre existian un margen de error.
    \item El metodo de minimos cuadrados nos ayuda a 
\end{itemize}
\section{Preguntas}
\begin{enumerate}
    \item \textbf{Del gafico obtenido en la primera parte hallar la velocidad instantanea en el punto C}
    \item \textbf{Que importancia tiene que las rectas se crucen anes o despues del eje de coordenadas o sea cuando $\Delta t=0$  ?}
    \item \textbf{Del grafico obtenido en las segunda parte, encontrar la aceleracio\'n.}
    \item \textbf{Comparar la velocidad instantanea en el punto C de la primera parte con la velocidad instantanea en funcion del tiempo de la segunda parte}
\end{enumerate}

\newpage

\section{Bibliografía}
\begin{itemize}
    \item  Serway. F\'isica. Editorial McGraw-Hill (1992).\\
    \item Tipler. Física. Editorial Revert\'e (1994). \\
    \item Taylor, J.R. (2014) Introducci\'on al Análisis de errores - reverte, Taylor 2 Muestra.pdf. Disponible en: https://www.reverte.com/media/reverte/files/book-attachment-3746.pdf (Accessed: April 14, 2023).

\end{itemize}
\end{document}