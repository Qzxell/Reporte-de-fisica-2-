\section{Objetivos}
\subsection{Objetivos generales}
Determinar experimentalmente la velocidad instantanea y la aceleracion de la rueda de Maxwell.
\subsection{objetivos espec\'ificos}
\begin{itemize}
    \item Ordenar los datos obtenidos en una tabla.
    \item Obtener distintas comparaciones y sacar conclusiones de los experimentos.
\end{itemize}
\section{C\'alculos}
\section{Resultados}
\section{Observaciones Experimentales}
\section{Conclusiones}
\begin{itemize}
    \item Podemos notar de los datos analizados varian un poco esto de debe a que al momento de tomar medidas siempre existian un margen de error.
    \item El metodo de minimos cuadrados nos ayuda a 
\end{itemize}
\section{Preguntas}
\begin{enumerate}
    \item \textbf{Del gafico obtenido en la primera parte hallar la velocidad instantanea en el punto C}
    \item \textbf{Que importancia tiene que las rectas se crucen anes o despues del eje de coordenadas o sea cuando $\Delta t=0$  ?}
    \item \textbf{Del grafico obtenido en las segunda parte, encontrar la aceleracio\'n.}
    \item \textbf{Comparar la velocidad instantanea en el punto C de la primera parte con la velocidad instantanea en funcion del tiempo de la segunda parte}
\end{enumerate}